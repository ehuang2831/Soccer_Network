One of our main novel contributions in this project would be assessing whether graph-based learning algorithms like Graph Convolutional Networks (GCNs) [2] perform better than other traditional learning algorithms in the context of prediction and analysis of sports games. 

There has been some related work in sports analytics conferences like the MIT Sloan Sports Analytics conference; for example, one paper [3] describes data-driven ghosting in soccer games which enables coaches and managers to ``scalably quantify, analyze and compare fine grained defensive behavior''. Their learning task was different than our proposed one here because they were more interested in analyzing and modeling player movements and defense styles, not predicting game statistics. Other related work includes an MIT Master's Thesis [4]  and a journal paper [5] which also use similar data from soccer games to infer passing patterns and styles [4], assess players' passing effectiveness, and predict shots [5]. However, to the best of our knowledge, none of these papers represent the data as a network graph and utilize the graph structure in their inferences, which is where our current work is situated. 


